\documentclass{beamer}

\usefonttheme[onlymath]{serif}
%\usepackage{blindtext}

%\usepackage{fontspec}
\newfontface\JuliaMonoRegular{JuliaMono-Regular.ttf}[Path = fonts/]
\newfontface\JuliaMonoBold{JuliaMono-Bold.ttf}[Path = fonts/]
\newfontface\JuliaMonoItalic{JuliaMono-RegularItalic.ttf}[Path = fonts/]
\setmonofont{JuliaMono-Medium.ttf}[Contextuals=Alternate, Path = fonts/]

\input{latex/julia_listings}
\usepackage{listings}

\lstdefinelanguage{JuliaLocal}{
    language = Julia, % inherit Julia lang. to add keywords
    morekeywords = [3]{@code_llvm, @benchmark}, % define more functions
    morekeywords = [2]{ClockTime}, % define more types and modules
}

\usepackage{amsmath}
\usepackage{hyperref}
\usepackage[export]{adjustbox}

\usetheme{Execushares}
\usefonttheme[onlymath]{serif}


\title{Extensible Software for Research}
\subtitle{principles and an example in julia}
\author{Maximilian Ernst}
\date{Center for Lifespan Psychology, Max Planck Institute for Human Development}

\setcounter{showSlideNumbers}{1}

\usepackage{svg}
\usepackage{graphicx}
\usepackage{tikz}
\usetikzlibrary{arrows}
\usetikzlibrary{positioning}
\usetikzlibrary{shapes}
\usetikzlibrary{fit}
\usetikzlibrary{backgrounds}

% logo
\logo{
\includegraphics[width=.2\textwidth]{figures/logo_blue.pdf}
}
\newcommand{\nologo}{\setbeamertemplate{logo}{}} % command to set the logo to nothing
\definecolor{mypurple}{RGB}{149, 88, 178}
\newcommand{\monob}[1]{\mbox{\texttt{\textcolor{mypurple}{#1}}}}

\newenvironment{wideitemize}{
    \itemize\addtolength{\itemsep}{15pt}\addtolength{\topsep}{10pt}}{\enditemize}

\newcommand{\A}{\mathbf{A}}
\newcommand{\I}{\mathbf{I}}
\newcommand{\F}{\mathbf{F}}
\newcommand{\OmegaM}{\mathbf{\Omega}}
\newcommand{\SigmaM}{\mathbf{\Sigma}}

\DeclareMathOperator{\prox}{prox}
\DeclareMathOperator*{\argmin}{arg\,min}

\begin{document}

	\setcounter{showProgressBar}{0}
	\setcounter{showSlideNumbers}{0}

	\frame{\titlepage}

	\setcounter{framenumber}{0}
	\setcounter{showProgressBar}{1}
	\setcounter{showSlideNumbers}{1}

    \begin{frame}
        \frametitle{Contents}
        \vspace{1cm}
        \begin{wideitemize}
            \item Why should you care?
        \end{wideitemize}
        \vspace{1cm}
        \begin{wideitemize}
            \item How do you get there?
        \end{wideitemize}
    \end{frame}

	\begin{frame}
        \frametitle{Research Software Personas}
        \begin{wideitemize}
            \item research software engineers
            \item statistics researcher
            \item applied user
        \end{wideitemize}
    \end{frame}

    \begin{frame}
        \frametitle{A day in the live of \ldots}
        a statistics researcher
        \vspace{0.8cm}
        \begin{wideitemize}
            \item work with a specific type of model
            \begin{itemize}
            \item linear regression, deep learning, \ldots
            \end{itemize}
            \item have an idea
            \item test it
            \item make it available to applied researchers
        \end{wideitemize}
    \end{frame}

    \begin{frame}
        \frametitle{Now we need software}
        \begin{wideitemize}
            \item to test $\to$ prototype
            \item to make it available $\to$ deploy
        \end{wideitemize}

    \vspace{0.5cm}

    What's the fastest way to get there?

    \vspace{0.5cm}

    existing software

    \vspace{0.5cm}

    \textbf{It would be nice if they could extend existing software}
    \end{frame}

    \begin{frame}
        \frametitle{But \ldots}
        \begin{wideitemize}
            \item \textbf{understand} 1000s of lines of code
            \item make \textbf{changes}, possibly breaking stuff
            \item get maintainers to \textbf{adopt} their changes
        \end{wideitemize}

        \vspace{1cm}

        \textbf{these hurdles are often too high!}
    \end{frame}

    \begin{frame}
        \frametitle{A day in the live of \ldots}
        \vspace{0.8cm}
        \begin{wideitemize}
            \item to test: barebone, minimal reimplementation
            \begin{itemize}
                \item waste of time
                \item not well tested
                \item hard to reproduce
            \end{itemize}
            \item to deploy: put their code on github
            \begin{itemize}
                \item bad user interface, no documentation
                \item missing features
                \item incompatible to existing software
            \end{itemize}
        \end{wideitemize}
    \end{frame}

    \begin{frame}
        \frametitle{My Experience}
        \vspace{0.8cm}
        \begin{wideitemize}
            \item from R $\to$ \raisebox{-.07\height}{\includegraphics[width=0.08\textwidth, valign=m]{figures/julia_logo.pdf}}
            \item most R packages are very hard to extend
            \item most \raisebox{-.07\height}{\includegraphics[width=0.08\textwidth, valign=m]{figures/julia_logo.pdf}} packages are very easy to extend
        \end{wideitemize}
    \end{frame}

    \begin{frame}
        \frametitle{Culture}
        \vspace{0.8cm}
        \begin{wideitemize}
            \item care about extensibility
            \item developer documentation
            \item assume their code is read
        \end{wideitemize}
    \end{frame}

    \begin{frame}
        \frametitle{Software Design}
        You need to be able to add new features...
        \vspace{0.8cm}
        \begin{wideitemize}
            \item without understanding existing code
            \item without changing existing code
            \item syntactical requirements need to be clear and easy communicable!
        \end{wideitemize}
    \end{frame}

    \begin{frame}
        \frametitle{Two modes of extension}
        \vspace{0.5cm}
        \includegraphics[width=0.5\textwidth]{figures/lorem_ipsum_red.pdf}%
        \includegraphics[width=0.5\textwidth]{figures/lorem_ipsum_green.pdf}
    \end{frame}

    \begin{frame}[fragile]
        \frametitle{Why julia?}
        In Julia, everything has a type:
        % show the type of a vey things
        \vspace{0.5cm}
        \begin{lstlisting}[language=JuliaLocal, style=julia]
a = 1.0
typeof(a) # Float64

b = "hello"
typeof(b) # String\end{lstlisting}
    \end{frame}

    \begin{frame}[fragile]
        \frametitle{Why julia?}
        You can define your own types:
        % define a walltime
        \vspace{0.5cm}
        \begin{lstlisting}[language=JuliaLocal, style=julia]
struct ClockTime{T}
    time::T
end

my_time = ClockTime(5.0) # ClockTime{Float64}(5.0)
my_time.time # 5.0\end{lstlisting}
    \end{frame}

    \begin{frame}[fragile]
        \frametitle{Why julia?}
        A function is a collection of methods:
        % show a base function with its methods
        % define multiplication for our type and/or its supertype
        \vspace{0.5cm}
        \begin{lstlisting}[language=JuliaLocal, style=julia]
typeof(1.0) # Float64
typeof(1) # Int64

@code_llvm 6.0*7.0
@code_llvm 6*7

methods(*)
\end{lstlisting}
    \end{frame}

        \begin{frame}[fragile]
        \frametitle{Why julia?}
        We can write our own addition:
        % show a base function with its methods
        % define multiplication for our type and/or its supertype
        \vspace{0.5cm}
        \begin{lstlisting}[language=JuliaLocal, style=julia, basicstyle=\scriptsize\JuliaMonoRegular]
import Base: +

function +(x::ClockTime{T}, y::ClockTime{T}) where{T}
    return ClockTime((x.time + y.time) % T(24))
end

my_time = ClockTime(11.2)
your_time = ClockTime(18.4)

our_time = my_time + your_time
\end{lstlisting}
    \end{frame}

    \begin{frame}[fragile]
        \frametitle{Goal achieved}
        We have just written extensible code.\\
        \vspace{0.5cm}
        \textit{I have memory problems, and I only care about full hours.}
        \vspace{0.5cm}
        \begin{lstlisting}[language=JuliaLocal, style=julia]
my_time = ClockTime(UInt8(5))
your_time = ClockTime(UInt8(8))

our_time = my_time + your_time\end{lstlisting}
    \end{frame}

        \begin{frame}[fragile]
        \frametitle{Why julia?}
        \textit{I want to multiply sparse matrices of clock times}
        \vspace{0.5cm}
        \begin{lstlisting}[language=JuliaLocal, style=julia, basicstyle=\scriptsize\JuliaMonoRegular]
using SparseArrays

import Base: zero, *

function *(x::T, y::ClockTime{T}) where T
    return ClockTime((x * y.time) % T(24))
end

zero(x::ClockTime{T}) where T = ClockTime(zero(T))
zero(::Type{ClockTime{T}}) where T = ClockTime(zero(T))
\end{lstlisting}
    \end{frame}

        \begin{frame}[fragile]
        \frametitle{Why julia?}
        Let's define some matrices!
        \begin{lstlisting}[language=JuliaLocal, style=julia, basicstyle=\scriptsize\JuliaMonoRegular]
a = zeros(ClockTime{Float64}, 20, 20)

a[1,1] = ClockTime(5.0)
a[1,2] = ClockTime(11.673)
a[6,9] = ClockTime(17.23)
a[16,4] = ClockTime(20.87)

a_sparse = sparse(a)

b = zeros(20, 20)
b[3,9] = 1
b[6,9] = sqrt(2)
b[19,1] = π
b[4,5] = ℯ

b_sparse = sparse(b)
\end{lstlisting}
    \end{frame}

            \begin{frame}[fragile]
        \frametitle{Why julia?}
        \vspace{0.5cm}
        \begin{lstlisting}[language=JuliaLocal, style=julia,
        basicstyle=\scriptsize\JuliaMonoRegular]
using BenchmarkTools

@benchmark b_sparse*a_sparse

BenchmarkTools.Trial: 10000 samples with 199 evaluations.
 Range (min … max):  422.864 ns …  12.295 μs
 Time  (median):     453.877 ns
 Time  (mean ± σ):   566.627 ns ± 648.451 ns
Memory estimate: 2.22 KiB

@benchmark b*a

BenchmarkTools.Trial: 10000 samples with 1 evaluation.
 Range (min … max):  56.683 μs …  1.323 ms
 Time  (median):     57.111 μs
 Time  (mean ± σ):   59.411 μs ± 18.922 μs
 Memory estimate: 23.75 KiB
\end{lstlisting}
    \end{frame}


    \begin{frame}{A few days in a methods researchers life}
        \begin{wideitemize}
        \item Do you have any ideas why this does not converge?
        \item Staring puzzled at the theory (should work?!).
        \item Staring very puzzled at the implementation in C++.
        \item Rinse and repeat for a couple of days and researchers.
        \end{wideitemize}
    \end{frame}

    \begin{frame}{An hour in our life}
      \begin{wideitemize}
        \item Look at the formula: $ridge(x, \lambda{}) = \lambda{}\sum^p_{j=1}x^2$
        \item Implement in Julia: \monob{ridge(x, λ) = λ ∗ sum(x.\^{}2))}
        \item add 30 lines of API (formal requirements)
        \item Enjoy.
      \end{wideitemize}
    \end{frame}

    \begin{frame}{Two hours in our life}
        \begin{wideitemize}
        \item Original simulation takes weeks on a dedicated workstation.
        \item Original simulation freezes our cluster due to bad parrelization.
        \item Simulation in Julia takes 2 hours on my laptop.
        \end{wideitemize}
    \end{frame}

    \begin{frame}{Why?}
        \begin{wideitemize}
        \item Some investments in extensibility
        \item division of labor: \begin{itemize}
            \item optimizing linear algebra is done by Intel
            \item numerical optimization is done by dedicated experts
            \item differentiation is automated
        \end{itemize}
        \item modern infrastructure
        \end{wideitemize}
    \end{frame}

    \begin{frame}{But why?}
    \centering \huge usability
    \end{frame}

    \begin{frame}{The effectiveness of usability}
      \begin{wideitemize}
        \item usability $\neq$ lazyness
        \item enables quick prototyping
        \item allows domain experts to contribute their expertise
        \item theorethical and technical development move in lockstep
      \end{wideitemize}
    \end{frame}

    \begin{frame}{How to improve usability?}
      \begin{enumerate}
        \item Extensible Software
        \item Dokumentation
        \item User Interface
      \end{enumerate}
    \end{frame}

    \begin{frame}{Dokumentation}
      \begin{wideitemize}
        \item Dokumentation for Users
        \item Dokumentation for Contributors
      \end{wideitemize}
    \end{frame}

    \begin{frame}{User Interface}
      \begin{wideitemize}
        \item Frictionless
        \item Connected to prior knowledge
      \end{wideitemize}
    \end{frame}

\end{document}

\documentclass{beamer}

\usefonttheme[onlymath]{serif}
%\usepackage{blindtext}

%\usepackage{fontspec}
\newfontface\JuliaMonoRegular{JuliaMono-Regular.ttf}[Path = fonts/]
\newfontface\JuliaMonoBold{JuliaMono-Bold.ttf}[Path = fonts/]
\newfontface\JuliaMonoItalic{JuliaMono-RegularItalic.ttf}[Path = fonts/]
\setmonofont{JuliaMono-Medium.ttf}[Contextuals=Alternate, Path = fonts/]

\input{latex/julia_listings}
\usepackage{listings}

\lstdefinelanguage{JuliaLocal}{
    language = Julia, % inherit Julia lang. to add keywords
    morekeywords = [3]{@code_llvm, @benchmark}, % define more functions
    morekeywords = [2]{ClockTime}, % define more types and modules
}

\usepackage{amsmath}
\usepackage{hyperref}
\usepackage[export]{adjustbox}

\usetheme{Execushares}
\usefonttheme[onlymath]{serif}


\title{Extensible Software for Research}
%\subtitle{principles and an example in julia}
\author{Aaron Peikert}
\date{Maximilian Ernst}

\setcounter{showSlideNumbers}{1}

\usepackage{svg}
\usepackage{graphicx}
\usepackage{tikz}
\usetikzlibrary{arrows}
\usetikzlibrary{positioning}
\usetikzlibrary{shapes}
\usetikzlibrary{fit}
\usetikzlibrary{backgrounds}

% logo
\logo{
\includegraphics[width=.2\textwidth]{figures/logo_blue.pdf}
}
\newcommand{\nologo}{\setbeamertemplate{logo}{}} % command to set the logo to nothing
\definecolor{mypurple}{RGB}{149, 88, 178}
\newcommand{\monob}[1]{\mbox{\texttt{\textcolor{mypurple}{#1}}}}

\newenvironment{wideitemize}{
    \itemize\addtolength{\itemsep}{15pt}\addtolength{\topsep}{10pt}}{\enditemize}

\newcommand{\A}{\mathbf{A}}
\newcommand{\I}{\mathbf{I}}
\newcommand{\F}{\mathbf{F}}
\newcommand{\OmegaM}{\mathbf{\Omega}}
\newcommand{\SigmaM}{\mathbf{\Sigma}}

\DeclareMathOperator{\prox}{prox}
\DeclareMathOperator*{\argmin}{arg\,min}

\begin{document}

	\setcounter{showProgressBar}{0}
	\setcounter{showSlideNumbers}{0}

	\frame{\titlepage}

	\setcounter{framenumber}{0}
	\setcounter{showProgressBar}{1}
	\setcounter{showSlideNumbers}{1}
	
    \begin{frame}
        \frametitle{Contents}
        \vspace{1cm}
        \begin{wideitemize}
            \item Principles of Extensible Research Software
        \end{wideitemize}
        \vspace{1cm}
        \begin{wideitemize}
            \item Application: StructuralEquationModels.jl
        \end{wideitemize}
    \end{frame}

	%\begin{frame}
    %    \frametitle{Who writes research software?}
    %    \begin{wideitemize}
    %        \item research software engineers
    %        \item research software developer
    %        \item applied user
    %    \end{wideitemize}
    %\end{frame}

    \begin{frame}
        \frametitle{A day in the live of \ldots}
        a statistical methods researcher\\
        \vspace{0.8cm}
        \begin{wideitemize}
            \item<2-> work with a specific type of model
            \begin{itemize}
            \item<2-> regression, structural equation models, deep learning, \ldots
            \end{itemize}
            \item<3-> have an idea
            \item<4-> test it
            \item<5-> make it available to applied researchers
        \end{wideitemize}
    \end{frame}

    \begin{frame}
        \frametitle{We need to write Software}
        \begin{wideitemize}
            \item to test $\to$ prototype
            \item to make it available $\to$ distribute
        \end{wideitemize}

    \vspace{0.5cm}

    \pause

    What's the fastest way to get there?

    \vspace{0.5cm}

    \pause

    We are already working with existing software.

    \vspace{0.5cm}

    \pause

    \textbf{It would be nice if we could extend existing software!}
    \end{frame}

    \begin{frame}
        \frametitle{But \ldots}
        \begin{wideitemize}
            \item<2-> \textbf{understand} 1000s of lines of code
            \item<3-> make \textbf{changes}, possibly breaking stuff
            \item<4-> get maintainers to \textbf{adopt} our changes
        \end{wideitemize}

        \vspace{1cm}

        \only<5>{\textbf{these hurdles are often too high!}}
    \end{frame}

    \begin{frame}
        \frametitle{A \only<1-2>{day}\only<3>{year} in the live of \ldots}
        \vspace{0.8cm}
        \begin{wideitemize}
            \item<2-> to test: barebone, minimal reimplementation
            \begin{itemize}
                \item waste of time
                \item not well tested
                \item hard to reproduce
                \item slow
            \end{itemize}
            \item<3-> to deploy: put their code on github
            \begin{itemize}
                \item bad user interface, no documentation
                \item missing features
                \item incompatible to existing software
            \end{itemize}
        \end{wideitemize}
    \end{frame}

    \begin{frame}
        \frametitle{My Experience}
        \vspace{1cm}
        \begin{wideitemize}
            \item from R $\to$ \raisebox{-.07\height}{\includegraphics[width=0.08\textwidth, valign=m]{figures/julia_logo.pdf}}
        \end{wideitemize}
    \end{frame}

    \begin{frame}
        \frametitle{Culture}
        \vspace{0.8cm}
        \begin{wideitemize}
            \item care about extensibility
            \item developer documentation
            \item assume their code is read
        \end{wideitemize}
    \end{frame}

    \begin{frame}
        \frametitle{An Example}
        Let's make a thought experiment...
        \vspace{0.8cm}
        \begin{wideitemize}
            \item The Encyclopedia
            \begin{itemize}
                \item add an entry
                \item some syntactical requirements\\
                {\small 300-400 words, alphabetical, \ldots}
            \end{itemize}
            \item Book
            \begin{itemize}
                \item a draft already exists
                \item add something everywhere it is applicable
            \end{itemize}
        \end{wideitemize}
    \end{frame}

    \begin{frame}
        \frametitle{An Example}\vspace{0.5cm}
        \includegraphics[width=0.5\textwidth]{figures/lorem_ipsum_red.pdf}%
        \includegraphics[width=0.5\textwidth]{figures/lorem_ipsum_green.pdf}
    \end{frame}

    \begin{frame}
        \frametitle{Software Design}
        \only<1>{Not all research articles can be encyclopedias, but maybe all research software can be...}
        \pause
        You need to be able to add new features...
        \vspace{0.8cm}
        \begin{wideitemize}
            \item without \textbf{understanding} existing code
            \item without \textbf{changing} existing code
            \item syntactical requirements need to be \textbf{clear} and \textbf{easy communicatable}!
        \end{wideitemize}
    \end{frame}

    \begin{frame}
        \frametitle{The Benefits}
        \vspace{0.5cm}
        \begin{wideitemize}
            \item Applied Researchers
            \begin{itemize}
                \item user interface
                \item better documentation
                \item faster availability of new features
                \item less bugs
                \item higher performance
            \end{itemize}
            \item Statistical Researchers
            \begin{itemize}
                \item no re-implementation $\to$ less time-consuming
                \item more users
                \item no software engineering skills needed
            \end{itemize}
            \item Maintainers
            \begin{itemize}
                \item changes are easier to integrate
            \end{itemize}
        \end{wideitemize}
    \end{frame}
    
    %%%%%%%%%%%%%%%%%%%%%%%%%% Aaron %%%%%%%%%%%%%%%%%%%%%%%%%%%%

    \begin{frame}{A few days in a methods researchers life}
        \begin{wideitemize}
        \item Do you have any ideas why this does not converge?
        \item Staring puzzled at the theory (should work?!).
        \item Staring very puzzled at the implementation in C++.
        \item Rinse and repeat for a couple of days and researchers.
        \end{wideitemize}
    \end{frame}

    \begin{frame}{An hour in our life}
      \begin{wideitemize}
        \item Look at the formula: $ridge(x, \lambda{}) = \lambda{}\sum^p_{j=1}x^2$
        \item Implement in Julia: \monob{ridge(x, λ) = λ ∗ sum(x.\^{}2))}
        \item add 30 lines of API (formal requirements)
        \item Enjoy.
      \end{wideitemize}
    \end{frame}

    \begin{frame}{Two hours in our life}
        \begin{wideitemize}
        \item Original simulation takes weeks on a dedicated workstation.
        \item Original simulation freezes our cluster due to poor parallelization.
        \item Simulation in Julia takes 2 hours on my laptop.
        \end{wideitemize}
    \end{frame}

    \begin{frame}{Why?}
        \begin{wideitemize}
        \item Some investments in extensibility
        \item division of labor: \begin{itemize}
            \item optimizing linear algebra is done by Intel
            \item numerical optimization is done by dedicated experts
            \item differentiation is automated
        \end{itemize}
        \item modern infrastructure
        \end{wideitemize}
    \end{frame}

    \begin{frame}{But why?}
    \centering \huge convenience
    \end{frame}
    
    \begin{frame}[c]{The leverage of extensible software}
    \centering
    \includegraphics[width=.8\textwidth]{figures/users.pdf}
    \end{frame}
    
    \begin{frame}{How to improve convenience?}
      \begin{enumerate}
        \item Extensible Software
        \item Dokumentation
        \item User Interface
      \end{enumerate}
    \end{frame}
    
  \begin{frame}[c]{Time is limited}
    \centering
    \includegraphics[width=.8\textwidth]{figures/difficulty.pdf}
  \end{frame}

    \begin{frame}{Dokumentation}
      \begin{wideitemize}
        \item Dokumentation for users
        \item Dokumentation for contributors/developers
      \end{wideitemize}
    \end{frame}

    \begin{frame}{User Interface}
      \begin{wideitemize}
        \item Frictionless
        \item Connected to prior knowledge
      \end{wideitemize}
    \end{frame}

\end{document}
